% "ModernCV" CV and Cover Letter
% LaTeX Template
% Version 1.1 (9/12/12)
%
% This template has been downloaded from:
% http://www.LaTeXTemplates.com
%
% Original author:
% Xavier Danaux (xdanaux@gmail.com)
%
% License:
% CC BY-NC-SA 3.0 (http://creativecommons.org/licenses/by-nc-sa/3.0/)
%
% Important note:
% This template requires the moderncv.cls and .sty files to be in the same 
% directory as this .tex file. These files provide the resume style and themes 
% used for structuring the document.

\hyphenation{Campinas}
\hyphenation{UNICAMP}


\documentclass[10pt, a4paper, roman]{moderncv} % Font sizes: 10, 11, or 12; paper sizes: a4paper, letterpaper, a5paper, legalpaper, executivepaper or landscape; font families: sans or roman


\usepackage[portuguese]{babel}
\usepackage[utf8]{inputenc}
\usepackage[T1]{fontenc}


\moderncvstyle{casual} % CV theme - options include: 'casual' (default), 'classic', 'oldstyle' and 'banking'

\moderncvcolor{blue} % CV color - options include: 'blue' (default), 'orange', 'green', 'red', 'purple', 'grey' and 'black'

%\usepackage{lipsum} % Used for inserting dummy 'Lorem ipsum' text into the template

\usepackage[scale=0.90]{geometry} % Reduce document margins
%\setlength{\hintscolumnwidth}{3cm} % Uncomment to change the width of the dates column
\setlength{\makecvtitlenamewidth}{10cm} % For the 'classic' style, uncomment to adjust the width of the space allocated to your name

\firstname{Guilherme} 
\familyname{Lucas}

% All information in this block is optional, comment out any lines you don't need
\title{Curriculum Vitae}

\mobile{+55 19 98279-1826}

\email{guilherme.slucas@gmail.com}

\begin{document}

\makecvtitle % Prints the CV title

\section{Experience}

\cventry{2021 -- Present}{Software Engineer II, Outlook Platform}{Microsoft}{São Paulo}{Brasil}
{
    \begin{itemize}
	    \item Designed, implemented with C\# and maintained in production a event based assistant that help users schedule their meetings.
	    \item Part of the on call rotation.
    \end{itemize}
}

\cventry{2019 -- 2021}{Software Engineer, Commercial Software Engineering}{Microsoft}{São Paulo}{Brasil}
{
    \begin{itemize}
	    \item Contributed to aks-engine, a open source Go project that Azure uses to deploy its managed kubernetes service. 
	    \item Develop the TypeScript library that supports a linter, cli and github actions. Linter became part of the guideline for the whole organization.
        \item Designed and implemented a notification system in Go for one of the biggest e-commerces in LATAM.
        \item Mentored five Cloud Solution Architects on infrastructure subjects.
        \item Created Powershell and Shell scripts to automate security rules for a covid-19 health bot deployed globally.
        \item Led the deployment process of a covid-19 Data API for one of the biggest health organizations in the world.
    \end{itemize}
}

\cventry{2018 -- 2019}{Software Engineering Intern}{Microsoft}{São Paulo}{Brasil}
{
    \begin{itemize}
        \item Created web client code with TypeScript for a Machine Learning Platform used by more than 300 data scientists.
        \item Developed Back-End code with Python, ASP.NET Core and Java Spring.
	    \item Design and implement Distributed Cloud Applications with Azure, Terraform, Docker and Kubernetes.
    \end{itemize}
}

\cventry{2017 -- 2018}{Data Science Intern}{CETAX}{São Paulo}{Brasil}
{
    \begin{itemize}
        \item Development of big data applications, writing API's in Python (Django) to obtain data and creating Machine Learning models, on Spark. 
        \item Helped the implementation of code review culture.
    \end{itemize}
}
\cventry{2016 -- 2017}{Undergraduate Research}{IBM - Unicamp}{Campinas}{Brasil}
{
    \begin{itemize}
        \item Helped to deploy and maintain Minicloud, a free open-source cloud for hundreds of Unicamp students.
        \item Developed \href{https://github.com/Guilhermeslucas/powergraph}{PowerGraph}, a Python monitoring software for a cloud server used by hundreds of people.
        \item Wrote several code to automate GNU/Linux machines deploys, in \href{https://github.com/Guilhermeslucas/Ansible-Code}{Ansible} and 
            \href{https://github.com/Guilhermeslucas/Dockerfiles}{Docker}.
        \item Interviewed and mentored three new students at the project.
    \end{itemize}
}
\subsection{Volunteer}
\cventry{2017 -- 2018}{Member}{LivreCamp - Unicamp}{Campinas}{Brasil}
{
    Worked as assistant and teacher on Shell Script and GNU/Linux courses.
}
\cventry{2014 -- 2015}{Organizing Committee}{Computing Week Unicamp (SECOMP)}{Campinas}{Brasil}
{
    Increased income by 85\% and conference participants by more than 50\%, reaching 280 people.
}
\cventry{2014 -- 2016}{Vice President}{CACo - Centro Acadêmico de Computação}{Campinas}{Brasil}
{
    Vice President of the students union for both computer science and engineering courses (nearly one thousand students).
}
\section{Education}

\cventry{2014 -- 2018}{Computer Engineering}{University of Campinas (UNICAMP)}{Campinas}{Brasil}
{
    \begin{itemize}
        \item Helped to implement a software that helps students choosing their subjects, with Python and Django.
        \item Won third place at Microsoft Hack the Campus Hackaton, with a chat bot that analyses university data.
        \item Volunteer Teaching Assistantant for Digital Circuits Subject.
    \end{itemize} 
}

\section{Languages}

\cvitemwithcomment{Native}{Portuguese}{}
\cvitemwithcomment{Fluent}{English}{}
\cvitemwithcomment{Basic}{Spanish}{}
\cvitemwithcomment{Basic}{French}{}

\section{Technical Skills}

\cvitem{Intermediate}{C/C++, Java, TypeScript, HTML, CSS, Docker, Ansible, Lisp}
\cvitem{Good}{Python, C\#, Go, Shell Script, Git, GNU/Linux Operational Systems, Azure, Kubernetes, Terraform}
\end{document}
